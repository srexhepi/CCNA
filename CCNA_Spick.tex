\documentclass[10pt, a4paper]{article}

%Set border of textbox
\usepackage{geometry}
	\geometry{
  		left=2cm,
  		right=2cm,
  		top=1cm,
  		bottom=1cm
  		}

\usepackage[utf8]{inputenc}
\usepackage{amsmath}
\usepackage{amsfonts}
\usepackage{amssymb}

\usepackage{multicol}

%Code
\usepackage{listings}
	%Define shell as basic font for listing
	\lstset{language=sh}
	
\usepackage{pdflscape}

\pagestyle{empty}

\begin{document}

\begin{center}
\huge CCNA most basic configurations
\end{center}
\vspace{1cm}

\begin{enumerate}

\item disable lookup \\ 
\lstinline|router# no ip domain-lookup|

\item set device name  \\ 
\lstinline|router# hostname R1|


\item assign privilege level access password \\ 
\lstinline|router# enable password class|

\item assing password to the VTY line \\ 
\lstinline| router# line vty 0 15| \\
\lstinline| router(config-line)# password cisco|


\item create message of the day (MOTD) \\ 
\lstinline|router# banner motd *This is the message of the day!*|


\item configure logging for the console \\
\lstinline|router# line con 0| \\
\lstinline|router(config-line)# logging synchronous|

\item encrypt plaintext passwords \\ 
\lstinline|router# service password-encryption|

\item enable unicast ipv6 routing \\ 
\lstinline| router# ipv6 unicast-routing|

\item copy running config to start config \\ 
\lstinline|router# Comming soon|
\end{enumerate}

\begin{landscape}
\begin{multicols*}{3}
	[
	\section{Summary of show commands in CISCO IOS}	
	This page summarizes all show commands which were often used in the labs. 
	It is not ordered by any topic of our CCNA course.
	]
\noindent
\begin{lstlisting}
show ip route
show ipv6 route
show ip interface brief
show ipv6 interface
\end{lstlisting}
	
\end{multicols*}
\end{landscape}

\end{document}